\section{The Transfer of Decision Making}

\par The one who makes a decision is always responsible for the consequences of their decision. Knowing that, their partner, if she is one of the devious women whom we are dealing with, will refuse to take a decisive stance in her relationship, preferring to stay in the ambiguity of contradictory behaviors with double meaning. The inconsistency in the way women treat us, but not in the way we, men, treat them (how sneaky of them!), is of high interest to them for keeping us in check while we sink into the hell of doubt.

\par In order to preserve uncertainty, and thus safeguard the mystery that perpetuates our confusions and doubts, devious women strictly refuse to make decisions that would impact the relationship in a definitive way. They never want to choose in a definitive manner between two paths, preferring to oscillate between both to enjoy the benefits of each of them, at the same time they try to wriggle out of the unpleasant consequences that are inherent to them. This is why their actions never define, in a decisive manner, if they want to be wives, lovers, casual dates, or cheaters, because they want to enjoy the benefits that each of these positions offer them without paying the corresponding price. When we protest, they try to lead us to make a decision that we may regret later, because then they can rub this fact in our face. The solution for such cases is this: to create a decisive situation that forces them to reveal in an unequivocal manner what they feel and how much they value us. Attempting to force them through arguments to define themselves is a waste of time. The best thing is to find a correct decision that we must make and whose result inevitably puts them in a decisive situation with no way out, forcing them to define themselves even if they do not want to. Then, we must communicate such a decision one-sidedly, totally avoiding an argument.

\par As a general rule, devious women are used to withdraw from a relationship without permanently disconnecting the man, because they want to keep him attached afterwards. For this purpose, they avoid to explicitly take their share of responsibility for the failure, implying that they are removing themselves due to our fault. They perform ingenious maneuvers to get out while keeping the sucker imprisoned. This is why they never have the courage to say to our face, in a clear, objective, and decisive manner, that they do not want us anymore, do not feel anything for us anymore, etc. They know if they do this, we will benefit from it, because we can push our lives in a different direction. It is a dishonest attitude, since it prevents us from turning the page and taking our own path. They want to be remembered later, want to feel and to be able to say that there is a rejected idiot who still loves them. The transfer of decision making to the other is a great mechanism for the satisfaction of this sadistic selfishness.