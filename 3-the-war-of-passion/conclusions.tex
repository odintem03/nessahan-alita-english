\unnumberedsection{Conclusions}

\par After reading this volume, I hope that the reader has noticed that we (the author of this book and the authors whom he was inspired by since this book's first volume) propose a peaceful resistance and inner tranquility as strategies to dismantle female perfidy. We propose a boycott of mind games, manipulation, tricks, and ruses. This boycott can only be carried out through the non-action, which will require from us the ability to accept everything and fairly return the consequences. The strategy that we propose is the same as Gandhi's: boycotting by silence and by refusal.

\par As the attentive reader may have already noticed, we do not argue in favor of negative feelings or misogynous sentimentality of any kind. We argue against passion and in favor of questioning and skepticism regarding the myth of the defenseless, fragile, weak, and harmless woman. Those who conclude that we are supporting negative feelings will be twisting this work. To be dispassionate is not to be selfish, arrogant, manipulative, vindictive, wrathful, or furious. Negative feelings are also passions.

\par When a hypothesis or a polemical idea causes discomfort, it must be refuted correctly, through demonstration of the failures in which it is based on, of its inner logical inconsistencies, and of its low explanatory power, instead of being simply depreciated. There is a difference between refuting a set of ideas and depreciating it. Mere depreciation is something subjective and vague.

\par Our conclusions do not extrapolate the field of romantic relationships, as many people mistakenly thought. Everything that was said here applies exclusively to the field of love and no other one. I hope that I have made it clear that our ideas are only valid for stable relationships and, therefore, are only intended for adult people.

\par The reader must also conclude that when dealing with female perfidy, we are only dealing with one aspect of the total human perfidy, which is much broader and takes qualitatively distinct forms in men and women. As was pointed out in this book, men also have their \enquote{shadow,} and there are women who do not let themselves be dominated by what is called their \enquote{obscure side,} thus truly expressing the face of the Sacred Feminine. We do not delve into these two aspects as a matter of focus, but there is nothing to prevent us from doing that in the future. So, generalizing would be absurd, since we can never meet all the people on the Earth. It must be understood that when we use the expressions \enquote{such women,} \enquote{these women,} \enquote{women,} \enquote{devious women,} \enquote{manipulative women,} etc., we are referring exclusively to insincere women who cheat in love, and not the other women. I will say nothing about the incidence of the profile outlined here in different countries, leaving this inquiry to the reader.