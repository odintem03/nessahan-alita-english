\section{The capricious judgments}

\par Female opinion, with respect to romantic relationships, is very often capricious, inconsequential, and lacking in logical-rational sense, which harms and infantilizes women.

\par Try to ask women why they have inconsistent conclusions and actions. Their answers will be illogical, confusing, subjective, and long-winded (from our masculine viewpoint, but not from theirs). Actually, they do not really know the answer, because they opt for complicated paths. The correct answer is the following: because they draw a conclusion from feelings. For women, what is right is what gives them pleasant feelings, and wrong is what displeases them emotionally. They are emotionally-driven beings. This is not to say that they are innocent and romantic as everyone thinks. They are actually very selfish, just like we are, but their selfishness is a sentimental and, more specifically, romantic one.

\par Because women have their judgment based on emotion, they develop capricious, absurd opinions such as, for example, that the worst men are the best and must be chased. None of them is able to properly explain why they do it. If we corner them into a discussion, they will defend themselves by trying to provoke all kinds of feelings in us: anger, mercy, shame, doubt, confusion, desire, fear, etc. They will use a high-pitched tone to try to scare us off, will scream, will cackle like a witch so as to provoke feelings of insignificance and ridiculousness, then will cry to make us feel sorry for them, and then will appeal to cynical, provocative words... These are ruses aimed at manipulating our emotions, and this never changes. Emotion is the field that they master and in which they move with ease.

\par It is noteworthy a very common tendency that happens in heated sexist arguments between people of the opposite gender, specially when they do not have a romantic relationship with each other. This tendency becomes visible when we openly criticize female ruses, and it consists of women attacking our masculinity by cynically qualifying us as homosexual. This is done in their most desperate moments, when all other attempts of emotional manipulation have failed. Usually, such attacks do work, disconcert, and confuse a heterosexual male, inducing him to worry about his self-image and to try to prove that he is not what the devious women is pretending to think of him. By running after this rubbish, the discussion is put aside. I have solved these interesting cases by simply unmasking women and saying that women who challenge the masculinity of a heterosexual male are actually challenging him to have sex with them, and are requesting sex. It usually works very well. Still, the ideal thing is to never argue.

\par When I say that female opinions are capricious, irrational, and irresponsible, many women get mad at me, but they should thank me, as I am denouncing something that harms them. If they made an effort to be more rational, without losing their superior emotionality and their delicacy, they would be less inconsequential, less futile, less inconsistent, would not be terrified of the truth, and would live better, because they would be less prone to hysterics and sadness. Unfortunately, our little playmates do not realize that the huge emptiness of sadness and boredom that they live in is directly linked to their foul play in love. Believing themselves very clever, they mistakenly assume that deception is the way to happiness.

\par Women's experimentally-verifiable sexual preference for men who do not love them and for promiscuous men is the undeniable proof that their conclusions are capricious and purely emotionally-driven. What is more, such preferences reward mediocrity and contribute to social degradation.

\par Their stubborn capricious opinions are only changed when the impact of their own mistakes hits them in their feelings, causing suffering. Unfortunately, the impacts are the consequences and, therefore, are only felt retrospectively, when it is usually too late. This is why it is no help to warn, to admonish, to advise, to fight, etc. and even less to argue or to polemicize.