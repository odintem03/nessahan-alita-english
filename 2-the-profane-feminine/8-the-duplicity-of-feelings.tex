\section{The duplicity of feelings}

\par Once you fall in love, you will become dependent on the presence of your beloved and will unceasingly chase after her so that your emotional pain is relieved. A devious women will then, purposefully and consciously, set up a barrier between you and her to prevent you from contacting her, and will keep you at a distance, lovesick. Her distancing, her impenetrable barrier, your lovesickness, and your chasing behavior present a directly proportional relation between each other. The more you fall in love, the more you will chase her. The more you chase her, the more impenetrable will be her barrier, the more distant she will become, and the longer her absence will last.

\par It is a curious phenomenon: a devious woman induces you to fall in love with her, but then stops your approach. By demanding the surrender of your heart, the devious woman is inducing you to chaser her and to flirt with her, for there is not passionate sentimentality without requirement for the presence and the proximity of the loved object. Contradictory and illogical? No, only cold and calculating. What is the logic or the point of such an apparently contradictory behavior? To keep you emotionally enslaved, forever imprisoned through passion, loving her for all eternity, without having your love reciprocated and without receiving anything in return. Unconscious? No, purposeful and calculated. What is her motivation? Absolute sentimental selfishness. How is this possible? Simply because we desperately desire women while they are almost indifferent to us.

\par How to reverse it? By fighting ourselves and developing traits that attract women. Which are these traits? I have comprehensively described them in my books.

\par Female contradictory behavior generates feelings of a contrary nature that struggle within us and tear our soul apart.

\par In the beginning of a relationship, when everything is a pretended bed of roses, our feelings of attachment and the like are triggered and reinforced by means of affectionate, caring, and loving attitudes of our partner. In this phase, a woman behaves like a saint, ignoring other males, etc. Later, when she verifies that we are trapped and infatuated, the degree of our dependence and emotional bond begins to be tested with provocative challenging attitudes. It is in this stage that we experience many conflicts, most of which coming from behaviors that subtly call into question her fidelity.

\par While we have not taken the bait, the female pretends to be exactly what we want. She acts like the perfect, wonderful, and divine lady of our most beautiful dreams. After we take the bait, however, her behavior changes little by little, and from paradise we fall into hell.

\par Her \enquote{innocent} attitudes of giving attention, showing sympathy, getting close, and taking care of other males annoy a man because they shake his conviction in the loyalty of feelings of his partner. They are used intentionally as a form of provocation. The problem is not in the promiscuous behavior of our partner, as feminists try to make it seem, but rather in the lack of transparency, in the uncertainty, confusion, and doubt that her behavior raises. If her attitude was clear and defined from the beginning, as in the case of a prostitute, there would be no problem. But, as we are rational, the annoyance of doubt, as Peirce says, corrodes us and provokes a great emotional suffering. We need defined situations. A relationship full of question marks and poorly explained facts cause a great torment, because we are left exclusively at the mercy of trust. As an irrational belief without logical basis is not our strength, we feel groundless. What infuriates us are contradictory behaviors. Subtle attitudes apparently without [weight] are seen by us as violent acts of disguised betrayals. Their innocence is apparent, because it is precisely the subtle attitudes without malice that facilitate cheating. Knowing this, devious women do precisely what we hate and they do it in a conscious and premeditated manner.

\par Not only in the field of fidelity does provocations happen. There are also attitudes that challenge and confront our feelings and values in many other fields.

\par As we are territorial and want to protect our genes, we need to continuously confirm through direct observation that our partner is absolutely faithful and keeps other males at a safe distance. For that reason, when a devious woman complains and says that we must trust her word, despite evidence of facts that create doubt instead of certainty, we feel that we are being deceived. The result is that we rage, rightly so, and begin to gradually develop hostile and negative feelings toward the one we only wanted to love. Such feelings are very bad for us, and, curiously, they make women happy for being proof that we suffer for what they did.

\par In the long run, there occurs a duplicity of feelings that blur the nature of the relationship: we nourish negative and, at the same time, positive feelings for the same person.

\par This simultaneous duplicity destroys us, because we cannot define what we feel anymore to polarize our attitudes. Positive feelings that we naively develop work as a break that does not allow us to antagonize them completely. Negative feelings prevents us from enjoying the fullness of the relationship. We are then left divided in two, split, loving and hating a same woman simultaneously. The bomb explodes inside us, in our heart. Our mistake, once again, consisted in we letting ourselves become intoxicated by the poison of passion. If we had resisted her fascination, her beauty, her charm, her delicacy, we would not have been pushed to the opposite extreme. Therefore, lust, attachment, admiration, longing, and other feelings alike are defects just as harmful as wrath, fury, hate, and jealousy. They all must be purged from our soul through analysis, understanding, assimilation, and prayer. I assert that, if you are an atheist, it will be harder to grow beyond yourself.

\par There is no other way out but to fall out of love. If you doubt that, try to fall madly in love and you will see the nefarious effects.

\par For the passionate man, I only see the following paths as possible:
\begin{enumerate}
	\item to commit suicide;
	\item to become homosexual;
	\item to be a conformed cuckold;
	\item to go mad;
	\item to transform yourself psychologically.
\end{enumerate}

\par I only recommend this last path.

\par The duplicity of feelings is tightly linked to the freeloading opportunistic nature of human beings. When they feel that they are losing us, they offer us their love, but when they feel they are winning us over, they give us indifference. As the master politician Machiavelli rightly taught us, human beings are more prone to take advantage than to reciprocate the love that they are given. Our partners are no exception to this law, and when they feel loved, they see such a fact as an opportunity to be exploited to the maximum, and not as an undeserved gift. The situation is even more serious to the extent that, still according to Machiavelli, we must not allow room for hatred, but only for apprehension. The solution is to keep reason on our side, so as to prevent women from hating us, and only \enquote{punish} them when they abuse our tolerance and trust. Females must feel loved but not too much, protected but not entirely, and must be afraid of a little pending punishment, whose intensity and limit are impossible to measure. The \enquote{punishments,} emotional ones in this case, must be fair, short, and impactful. The benefits and rewards for her good behavior must be distributed slowly for a long time, so that they are enjoyed and remembered for a long time.

\par If we let a woman's attempt to deceive us pass unchallenged after it was discovered, we will bring discredit on ourselves. Instead of recognizing the value of our noble reasons that motivate us to forgive them (understanding, forgiveness, mercy, compassion), females will take us for weak men, because their gold-digging mind is unable to understand the value of noble feelings. They will reciprocate our forgiveness with freeloading opportunism and not with gratitude, and will tell themselves, \enquote{What a weak man! He has no guts to stop me and passively allows me to take advantage of his trust and abuse his tolerance.} On the other hand, if we accurately \enquote{punish} them when it comes to their feelings, making them suffer as they do to us, giving them back everything at the same time that we fling open their malfeasance in an explicit manner, they begin to admire us on their inside, even if they cry, complain, and protest. It follows, thus, that the most cruel and vengeful men are more admired than merciful gracious ones, unfortunately. Once again we can see that lovestruck men disgrace themselves.

\par Notice that a woman's provocative behavior (lying, attempting to deceive, to manipulate, to double-cross, etc.) is dissimulated and takes on an innocent appearance. It also tends to show up precisely when the atmosphere between the couple is wonderful, because females have no respect for the wellness of relationships. Actually, a wonderful sense of wellness is seen by them as an opportunity to be taken advantage of, that is, it is a sign that the most favorable time for double-crossing us has come, for it is the moment that we are being most malleable and \enquote{nice.} They feel that they must take advantage of this moment as soon as possible. This is why they suddenly ruin a harmonious coexistence. They tend to surprise us by ruining our good times with negative attitudes when we are being friendly. They raise our expectations for certain attitudes and surprise us with contrary attitudes, thus rendering our duplicity of feelings unavoidable. Our feelings, sincerity, and trust are seen as objects to be used without the slightest regard. Hence the importance of armoring ourselves at this level.

% [...]

% Note: corrected to Pat Mills (not Path Mills)
\par The paradoxical female conduct that evokes conflicting feelings reminds me of what the Earth Goddess, who represents the Eternal Feminine, said to Sláine, the Celtic hero by Pat Mills:

\begin{center}
\textit{\enquote{%
Sometimes I am your mother and I hold you...\\%
\hfill \break%
Sometimes I am your sister and I help you...\\%
\hfill \break%
And sometimes I am the lover who stabs you in the back!%
}}
\end{center}

\par The woman whom we fall in love with can be any of them.